% Options for packages loaded elsewhere
\PassOptionsToPackage{unicode}{hyperref}
\PassOptionsToPackage{hyphens}{url}
\PassOptionsToPackage{dvipsnames,svgnames,x11names}{xcolor}
%
\documentclass[
  11pt,
  justified]{article}
\usepackage{amsmath,amssymb}
\usepackage{iftex}
\ifPDFTeX
  \usepackage[T1]{fontenc}
  \usepackage[utf8]{inputenc}
  \usepackage{textcomp} % provide euro and other symbols
\else % if luatex or xetex
  \usepackage{unicode-math} % this also loads fontspec
  \defaultfontfeatures{Scale=MatchLowercase}
  \defaultfontfeatures[\rmfamily]{Ligatures=TeX,Scale=1}
\fi
\usepackage{lmodern}
\ifPDFTeX\else
  % xetex/luatex font selection
  \setmainfont[]{Times New Roman}
\fi
% Use upquote if available, for straight quotes in verbatim environments
\IfFileExists{upquote.sty}{\usepackage{upquote}}{}
\IfFileExists{microtype.sty}{% use microtype if available
  \usepackage[]{microtype}
  \UseMicrotypeSet[protrusion]{basicmath} % disable protrusion for tt fonts
}{}
\makeatletter
\@ifundefined{KOMAClassName}{% if non-KOMA class
  \IfFileExists{parskip.sty}{%
    \usepackage{parskip}
  }{% else
    \setlength{\parindent}{0pt}
    \setlength{\parskip}{6pt plus 2pt minus 1pt}}
}{% if KOMA class
  \KOMAoptions{parskip=half}}
\makeatother
\usepackage{xcolor}
\usepackage[margin=1in,margin=1in]{geometry}
\usepackage{graphicx}
\makeatletter
\def\maxwidth{\ifdim\Gin@nat@width>\linewidth\linewidth\else\Gin@nat@width\fi}
\def\maxheight{\ifdim\Gin@nat@height>\textheight\textheight\else\Gin@nat@height\fi}
\makeatother
% Scale images if necessary, so that they will not overflow the page
% margins by default, and it is still possible to overwrite the defaults
% using explicit options in \includegraphics[width, height, ...]{}
\setkeys{Gin}{width=\maxwidth,height=\maxheight,keepaspectratio}
% Set default figure placement to htbp
\makeatletter
\def\fps@figure{htbp}
\makeatother
\setlength{\emergencystretch}{3em} % prevent overfull lines
\providecommand{\tightlist}{%
  \setlength{\itemsep}{0pt}\setlength{\parskip}{0pt}}
\setcounter{secnumdepth}{-\maxdimen} % remove section numbering
\newlength{\cslhangindent}
\setlength{\cslhangindent}{1.5em}
\newlength{\csllabelwidth}
\setlength{\csllabelwidth}{3em}
\newlength{\cslentryspacingunit} % times entry-spacing
\setlength{\cslentryspacingunit}{\parskip}
\newenvironment{CSLReferences}[2] % #1 hanging-ident, #2 entry spacing
 {% don't indent paragraphs
  \setlength{\parindent}{0pt}
  % turn on hanging indent if param 1 is 1
  \ifodd #1
  \let\oldpar\par
  \def\par{\hangindent=\cslhangindent\oldpar}
  \fi
  % set entry spacing
  \setlength{\parskip}{#2\cslentryspacingunit}
 }%
 {}
\usepackage{calc}
\newcommand{\CSLBlock}[1]{#1\hfill\break}
\newcommand{\CSLLeftMargin}[1]{\parbox[t]{\csllabelwidth}{#1}}
\newcommand{\CSLRightInline}[1]{\parbox[t]{\linewidth - \csllabelwidth}{#1}\break}
\newcommand{\CSLIndent}[1]{\hspace{\cslhangindent}#1}
\usepackage{booktabs}
\usepackage{longtable}
\usepackage{array}
\usepackage{multirow}
\usepackage{wrapfig}
\usepackage{float}
\usepackage{colortbl}
\usepackage{pdflscape}
\usepackage{tabu}
\usepackage{threeparttable}
\usepackage{threeparttablex}
\usepackage[normalem]{ulem}
\usepackage{makecell}
\usepackage{xcolor}
\ifLuaTeX
  \usepackage{selnolig}  % disable illegal ligatures
\fi
\IfFileExists{bookmark.sty}{\usepackage{bookmark}}{\usepackage{hyperref}}
\IfFileExists{xurl.sty}{\usepackage{xurl}}{} % add URL line breaks if available
\urlstyle{same}
\hypersetup{
  colorlinks=true,
  linkcolor={Maroon},
  filecolor={Maroon},
  citecolor={Blue},
  urlcolor={black},
  pdfcreator={LaTeX via pandoc}}

\author{}
\date{\vspace{-2.5em}}

\begin{document}

\begin{center}
  \Huge{\textbf{Decentralized Dynamic General Equilibrium Modeling\\}}
  \vspace*{0.5cm}
  \Large{\textbf{Tshidiso Mofokeng\\}}
  \par
  \vspace*{2cm}
\end{center}

\begin{center}
\textbf{Abstract}
\end{center}

This paper investigates contemporary macroeconomics by developing a
decentralized dynamic general equilibrium model in MATLAB using Dynare.
It focuses on micro-founded technique for modeling real-world business
cycles and addresses constraints to understanding and predict economic
phenomena. The paper further builds a benchmark DGE model, incorporating
a utility function with two state variables, and presents Impulse
Response Functions in linearised and log-linearised forms. This offers
insights into economic implications of a shock to technology on various
macroeconomic variables. The findings were that output, consumption,
investment, labour, the rental rate of capital and wages increased in
response to a 1\% shock in technology. Whereas, the marginal utility of
consumption and leisure subsided in response to a 1\% shock in
technology.

\hypertarget{introduction}{%
\section{Introduction}\label{introduction}}

Contemporary macroeconomics has placed more focus on making use of
micro-founded methodology to model relevant real-world situations
business cycles. The key ingredient in macro modelling is solving a
problem under constraints to address a particular problem at hand. In so
doing, a macro model helps us to understand a particular problem in the
past or present and to make predictions about the future and also to
carry counter factual experiments.

The benchmark model that is built reflects the way current macroeconomic
analysis is done at the frontier of macroeconomic research and at
central banks and government. This paper sets out to build a
decentralised dynamic general equilibrium model (DGE). The primary
software used to model outcomes in this study is Dynare, of which runs
on MATLAB. Moreover, it is well accepted in the literature as the main
program for dynamic stochastic general equilibrium modeling.

The paper firstly sets up the consumers maximum lifetime utility
function subject to a budget constraint. Moreover, we include not one
but two state variables, that is, capital stock as well as bonds. The
underlying assumption of building our model is that along the balanced
growth path, the economy does not grow or rather the economy grows at
zero. Thereafter, we solve for the steady states of all variables in our
model. Subsequently, the model is calibrated which entails finding
unique set of model parameters that provide a good description of the
system behaviour. Lastly, we provide the Impulse Response Functions in
levels, logs as well as the log-linearised form and offer a discussion
on the economic meaning.

\newpage

\hypertarget{section-2}{%
\section{Section 2}\label{section-2}}

Setting up the consumer's infinite utility maximization problem in
recursive form using bond as an additional state variable. We obtain:

\[
V(k_{t}, b_{t}) = \sum_{t=s}^{\infty} \max_{\{c_t, k_{t+1}, b_{t+1}\}} [u(c_t) + \beta(k_{t+1}, b_{t+1})]
\]

Since we are given the budget constraint, we insert in into our
consumer's infinite maximization problem to find,

\[
V(k_t, b_t) =  \max_{\{l_t, k_{t+1}, b_{t+1}\}} \sum_{t=s}^{\infty} u [w_tl_t + r_t^k k_t - k_{t+1} + (1+\delta_k)k_t - b_{t+1}  + (1+r_t)b_t + \pi_t, 1-l_t] +  \beta V(k_{t+1}, b_{t+1})
\]

Now given the current period utility be of log form as
\(u(c_t, x_t) = ln(c_t) + \alpha ln(x_t)\). The two margins w.r.t
\(c_t\) and \(x_t\) can be written as,

\begin{align}
\frac{\partial u(c_t, x_t)}{\partial c_t} = \frac{1}{c_t} = \lambda_t, \tag{marginal utility of consumption} \label{eq:mu_cons}
\end{align}

and

\[
\frac{\partial u(c_t, x_t)}{\partial c_t} = \frac{\alpha}{x_t}
\]

\hypertarget{section-3}{%
\section{Section 3}\label{section-3}}

We decentralize and solve for two problems. Firstly for the consumer
problem, then using the results from the consumer problem, we
subsequently solve for the firm problem. Decentralising the consumer and
firm problem will allow prices to be made implicit. Taking the first
order condition w.r.t \(l_t\) we get,

\[
\frac{\partial u(c_t, x_t)}{\partial c_t}  w_t + \frac{\partial u(c_t, x_t)}{\partial x_t}  (-1) = 0
\]

Now substituting the two margins from section 2 back into the first
order conditions of the consumer's infinite maximization problem w.r.t
\(l_t\) we obtain,

\[
\frac{1}{c_t} w_t + \frac{\alpha}{x_t} = 0
\]

\[
w_t = \frac{\alpha c_t}{x_t}
\]

\[
\implies \frac{\frac{\alpha}{x_t}}{\frac{1}{c_t}}
\]

\[
\implies \frac{\alpha \frac{\partial u(c_t, x_t)}{\partial x_t}}{ \frac{\partial u(c_{t+1}, x_{t+1})}{\partial c_{t}}}
\]

This is the marginal product of labour otherwise known as the real wage
or the shadow price. Now taking the first order condition of the
consumer's infinite maximization problem w.r.t \(k_{t+1}\) we get,

\begin{align}
\frac{\partial u(c_t, x_t)}{\partial c_t} (-1) + \beta \frac{\partial V(k_{t+1}, b_{t+1})}{\partial k_{t+1}} = 0, \tag{1} \label{eq:one}
\end{align}

But,

\begin{align}
\frac{\partial V(k_{t}, b_{t})}{\partial k_{t}} = \frac{\partial u(c_t, x_t)}{\partial c_t} [1 + r_t^k - \delta_k], \tag{by envelope condition}
\end{align}

\begin{align}
\implies \frac{\partial V(k_{t}, b_{t})}{\partial k_{t}} = \frac{1}{c_t} [1 + r_t^k - \delta_k], \tag{using the marginal utility (MU) of consumption}
\end{align}

Hence advancing t to t+1 gives,

\[
\frac{\partial V(k_{t+1}, b_{t+1})}{\partial k_{t+1}} =  \frac{1}{c_{t+1}} [1 + r_{t+1}^k - \delta_k]
\]

Substituting Euler equation also known as the intertemporal margin back
into equation \ref{eq:one} yields,

\[
\frac{1}{c_t} (-1) + \beta \frac{1}{c_{t+1}} [1 + r_{t+1}^k - \delta_k] = 0
\]

Simplifying we obtain,

\begin{align}
\beta \frac{1}{c_{t+1}} [1 + r_{t+1}^k - \delta_k] = \frac{1}{c_t}, \tag{Euler equation}
\end{align}

\[
\beta [1 + r_{t+1}^k - \delta_k] = \frac{c_{t+1}}{c_t} = \frac{\frac{\partial u(c_t, x_t)}{\partial c_t}}{\frac{\partial u(c_{t+1}, x_{t+1})}{\partial c_{t+1}}}
\]

This is the MRS between consumption today and consumption tomorrow.
Hereafter, we take the first order condition of the consumer's infinite
maximization problem w.r.t \(b_{t+1}\), of which we obtain,

\begin{align}
\frac{\partial u(c_t, x_t)}{\partial c_t} . (-1) + \beta \frac{\partial V(k_{t+1}, b_{t+1})}{\partial b_{t+1}} = 0, \tag{2} \label{eq:two}
\end{align}

But again,

\begin{align}
\frac{\partial V(k_{t}, b_{t})}{\partial b_{t}} = \frac{\partial u(c_t, x_t)}{\partial c_t} [1 + r_t^k] \tag{by envelope condition}
\end{align}

\begin{align}
\implies \frac{\partial V(k_{t}, b_{t})}{\partial b_{t}} = \frac{1}{ c_t} [1 + r_t^k] \tag{using MU of consumption}
\end{align}

Hence advancing t to t+1 gives,

\begin{align}
\frac{\partial V(k_{t+1}, b_{t+1})}{\partial b_{t+1}} =  \frac{1}{c_{t+1}} [1 + r_{t+1}^k], \tag{3} \label{eq:three}
\end{align}

Substituting equation \ref{eq:three} into equation \ref{eq:two} then
yields,

\[
\frac{1}{c_t} (-1) + \beta \frac{1}{c_{t+1}} [1 + r_{t+1}^k] = 0
\]

Simplifying we obtain,

\[
\beta [1 + r_{t+1}^k] = \frac{c_{t+1}}{c_t}  
\]

\[
\implies \frac{\frac{\partial u(c_t, x_t)}{\partial c_t}}{\frac{\partial u(c_{t+1}, x_{t+1})}{\partial c_{t+1}}}
\]

This is also the MRS between consumption today and consumption tomorrow.
Hence from above, we note three consumer equilibrium conditions,

\[
w_t = \frac{\alpha c_t}{x_t}
\]

\[
\frac{c_{t+1}}{c_t}  = \beta [1 + r_{t+1}^k] 
\]

\[
\implies \frac{1+r_{t+1}^k}{1+\rho}
\]

where the discount factor \(\beta = \frac{1}{1+\rho}\).

\[
\frac{c_{t+1}}{c_t}=\beta[1+r_{t+1}^k-\delta_k]  
\]

\[
\implies \frac{1+r_{t+1}^k-\delta_k}{1+\rho}
\]

Now consider the firm problem. We are given the production function in
terms of the Cobb-Douglas form as
\(y_t = A_g l_t^\gamma k_t^{1-\gamma}\). We want to maximize profit in
the firms problem subject to production. Hence our firm problem becomes,

\[
V(k_t, b_t) =  \max_{\{l_t, k_{t}\}} \pi_t  A_g l_t^\gamma k_t^{1-\gamma} - w_t l_t - r_t^k k_t  
\]

Our first order conditions w.r.t \(l_t\) yield,

\[
\frac{\partial V(k_t, b_t)}{\partial l_t} = \gamma A_g l_t^{\gamma - 1} k_t^{1-\gamma} - w_t = 0 
\]

\[
\implies w_t = \gamma A_g l_t^{\gamma - 1} k_t^{1-\gamma}
\]

Simplifying this yields,

\[
= \frac{\gamma A_g k_t^{1-\gamma}} {l_t^{1-\gamma}}
\]

\begin{align}
= \gamma A_g [\frac{k_t}{l_t}]^{1-\gamma}, \tag{real wage} \label{eq:real_wage}
\end{align}

Subsequently, the first order condition w.r.t \(k_t\) yields,

\[
\frac{\partial V(k_t, b_t)}{\partial k_t} = \gamma A_g l_t^{\gamma} k_t^{-\gamma} - r_t^k = 0 
\]

\[
\implies r_t^k = (1-\gamma) A_g l_t^{\gamma} k_t^{-\gamma} 
\]

Simplifying gives us,

\[
= (1-\gamma) \frac{\gamma A_g k_t^{- \gamma}} {l_t^{- \gamma}}
\]

\begin{align}
= (1-\gamma) A_g [\frac{k_t}{l_t}]^{- \gamma}\tag{real interest rate} \label{eq:real_interest}
\end{align}

These are our firm equilibrium conditions, that is the real wage and the
real interest rate. Overall, taking the first order conditions and
solving for the respective variables. We find that the consumer has
three equilibrium conditions, on the other hand, the firm has two
equilibrium conditions. As a result, we note that decentralising the
consumer and firm problem makes prices explicitly known.

\hypertarget{section-4}{%
\section{Section 4}\label{section-4}}

In this section aims to solve the steady state of all the variables in
the model. Assuming net supply/demand of bond (debt) is zero and that
productivity (\(\bar A = 1\)). The firms profit is zero after paying out
wages and rents. We are given that consumption and investment are
expected to exhaust output, that is \(y_t = c_t + i_t\). Since savings
equals investment in equilibrium we can then rewrite the law of motion
of capital w.r.t interest as \(i_t = k_{t+1} - (1-\delta_k)k_t\).

Furthermore, in the case of zero growth, all variables are stationary in
the equilibrium. Assuming zero growth would in one case mean that
consumption is not growing and so \(\frac{c_{t+1}}{c_t} = 1\). Hence,

\[
1 = \frac{1 + r_{t}^k - \delta_k}{1+\rho}
\]

\[
1+\rho = 1 + r_{t}^k - \delta_k
\]

\[
r_{t}^k = \rho + \delta_k
\]

Returning to the firm equilibrium condition, we found the marginal
product of capital to be,

\[
\rho + \delta_k = r_{t}^k = (1-\gamma) A_g [\frac{k_t}{l_t}]^{- \gamma}
\]

Simplifying we obtain,

\[
\frac{\rho + \delta_k}{(1-\gamma) A_g} = [\frac{k_t}{l_t}]^{ - \gamma}
\]

\begin{align}
\frac{k_t}{l_t} = [\frac{\rho + \delta_k}{(1-\gamma) A_g}]^{-\frac{1}{\gamma}},  \tag{capital to labour ratio} \label{eq:kl}
\end{align}

Returning to the budget constraint, recall it was given as,

\[
c_t = w_tl_t + r_t^k k_t - k_{t+1} + (1+\delta_k)k_t - b_{t+1}  + (1+r_t)b_t + \pi_t
\]

Assuming zero growth implies no change in the capital stock so that
\(\frac{k_{t+1}}{k_t} = 1 \implies k_t = k_{t+1}\) and
\(\frac{b_{t+1}}{b_t} = 1 \implies b_t = b_{t+1}\) and the firm makes no
profit. Moreover, in steady state for bonds we have \(\bar b = 0\). We
note that capital is constant in equilibrium and so from the law of
motion,

\[
i_t = \delta_k k_t
\]

Furthermore, the consumption demanded now becomes,

\[
c_t = w_tl_t + r_t^k k_t - k_{t} + (1+\delta_k)k_t - b_{t}
\]

This simplify's to,

\[
= w_tl_t + r_t^k k_t - \delta_kk_t 
\]

Note from the consumer equilibrium condition, our \(MRS_{c_t, x_t}\) in
terms of leisure was obtained as,

\[
w_t = \frac{\alpha c_t}{x_t}
\]

Re-writing this finally yields,

\begin{align}
x_t = \frac{\alpha c_t}{w_t} \tag{4} \label{eq:four}
\end{align}

Since we given that \(y_t = c_t + i_t\), re-writing this in terms of
consumption yields \(c_t = y_t - i_t\). Since we know our production
function and our law of motion in equilibrium, we can represent
consumption as \(c_t\) =
\(A_g l_t^{\gamma} k_t^{1-\gamma} - \delta_k k\). Furthermore, taking
the \ref{eq:real_wage} equation from the first order condition of the
firm is and solving for leisure in equation \ref{eq:four} yields,

\[
x_t = \frac{\alpha [A_g [\frac{l_t}{k_t}]^{\gamma} k_t - \delta_k k_t]}{ \gamma A_g [\frac{k_t}{l_t}]^{1- \gamma}}
\]

The time constraint in terms of leisure can be written as
\(x_t = 1- l_t\),

\[
1 - l_t = \frac{\alpha [A_g [\frac{l_t}{k_t}]^{\gamma} - \delta_k ] k_t}{ \gamma A_g [\frac{k_t}{l_t}]^{1-\gamma}}
\]

\[
l_t = 1 - \frac{\alpha [A_g [\frac{l_t}{k_t}]^{\gamma} - \delta_k] k_t}{ \gamma A_g [\frac{k_t}{l_t}]^{1-\gamma}}  
\]

Let \[
\Gamma k_t = \frac{\alpha [A_g [\frac{l_t}{k_t}]^{\gamma} - \delta_k]k_t}{ \gamma A_g [\frac{k_t}{l_t}]^{1-\gamma}}
\]

where
\(\Gamma = \frac{\alpha [A_g [\frac{l_t}{k_t}]^{\gamma} - \delta_k]}{ \gamma A_g [\frac{k_t}{l_t}]^{1-\gamma}}\).
Recall the \ref{eq:kl} equation. Subsequently, simplifying w.r.t \(l_t\)
gives us,

\[
l_t = \frac{k_t}{[\frac{\rho + \delta_k}{(1-\gamma) A_g}]^{-\frac{1}{\gamma}}}
\]

Let

\[
\psi = [\frac{\rho + \delta_k}{(1-\gamma) A_g}]^{-\frac{1}{\gamma}}
\]

Then we can write \(l_t\) as

\[
l_t = \psi k_t
\] Returning to the time constraint in terms of leisure, we can now
write this as,

\[
1 - \Gamma k_t =  \psi k_t
\]

Solving for k yields,

\[
1 = \psi k_t + \Gamma k_t
\]

\[
k_t = \frac{1}{\psi + \Gamma} 
\]

\[
\implies \frac{1}{ [\frac{\rho + \delta_k}{(1-\gamma) A_g}]^{-\frac{1}{\gamma}} + \alpha \frac{ A_g (\frac{l_t}{k_t})^{\gamma} - \delta_k}{ \gamma A_g (\frac{k_t}{l_t})^{1-\gamma}}} 
\]

\[
\implies \frac{\gamma A_g (\frac{k_t}{l_t})^{1-\gamma}}{\gamma A_g (\frac{k_t}{l_t})^{1-\gamma} [\frac{\rho + \delta_k}{(1-\gamma) A_g}]^{-\frac{1}{\gamma}} + \alpha [A_g (\frac{l_t}{k_t})^{\gamma} - \delta_k]}
\]

Thus in steady state, after having solved for all the variables in the
model, we find 10 variables with 10 equations as:

\begin{align}
\frac{k_t}{l_t} = [\frac{\rho + \delta_k}{(1-\gamma) A_g}]^{-\frac{1}{\gamma}}  \tag{Capital to labour ratio}
\end{align}

\begin{align}
w_t  = \gamma A_g [\frac{k_t}{l_t}]^{1-\gamma} \tag{Real wage} 
\end{align}

\begin{align}
r_t = (1-\gamma) A_g [\frac{k_t}{l_t}]^{- \gamma} \tag{Real interest rate}
\end{align}

\begin{align}
l_t = \frac{k_t}{[\frac{\rho + \delta_k}{(1-\gamma) A_g}]^{-\frac{1}{\gamma}}} \tag{Labour}
\end{align}

\begin{align}
k_t = \frac{\gamma A_g (\frac{k_t}{l_t})^{1-\gamma}}{\gamma A_g (\frac{k_t}{l_t})^{1-\gamma} [\frac{\rho + \delta_k}{(1-\gamma) A_g}]^{-\frac{1}{\gamma}} + \alpha [A_g (\frac{l_t}{k_t})^{\gamma} - \delta_k]} \tag{Captial}
\end{align}

\begin{align}
i_t = \delta_k k_t \tag{Investment} 
\end{align}

\begin{align}
y_t = A_g [\frac{k_t}{l_t}]^{1-\gamma} k_t \tag{Output}
\end{align}

\begin{align}
c_t = A_g [\frac{k_t}{l_t}]^{1-\gamma} k_t - \delta_k k_t \tag{Consumption}
\end{align}

\begin{align}
\lambda = c^{-1} \tag{Marginal utility of consumption} 
\end{align}

\begin{align}
x_t = \frac{\alpha [A_g [\frac{l_t}{k_t}]^{\gamma} k_t - \delta_k k_t]}{ \gamma A_g [\frac{k_t}{l_t}]^{1- \gamma}} \tag{Leisure}
\end{align}

\hypertarget{section-5}{%
\section{Section 5}\label{section-5}}

In this section we calibrate the parameters pertaining to the technology
process, namely the persistence in the process \(\rho_{\alpha}\) and the
volatility of the shock or its standard deviation \(\sigma_\alpha\). We
assume that the technology process follows a zero mean autoregressive
order one \(AR(1)\) process in the form:

\[ log(A_t) = \rho_{\alpha}log(A_t) + \varepsilon _t\]

Harris (\protect\hyperlink{ref-harris2015}{2015}) proposed an approach
to finding \(\rho_{\alpha}\) and \(\sigma_\alpha\), in the case when the
technological process is not observable or there is not any data on it.
Since we have data on the real gross domestic product, the labour as
well as the capital which is \(GDP_t, l_t\) and \(k_t\) respectively.

Let the production function be of Cobb-Douglas form: \[
yt = A_t l_t^\gamma k_t^{1-\gamma}
\]

where \(A, l_t\) and \(k_t\) are the productivity process, labour and
capital respectively. We begin by adding a new variable to the original
dataset called \(A_t\) which is a vector of 288 zeros. Then we add
another new variable \(\gamma = \frac{1}{3}\) which is the labour
elasticity that is given or already known. Thereafter, we log our
\(GDP_t, l_t\) and \(k_t\). We are now in position to build a time
series for the technological process that directly estimates the process
using the Solow residual (\protect\hyperlink{ref-harris2015}{Harris
2015}). Solow (\protect\hyperlink{ref-solow1957technical}{1957})
recommended, measuring the portion of growth that can be attributable to
technological development as output growth that remains unexplained
after accounting for input growth. From the production function, solving
for \(A_t\) returns,

\[
A_t = \frac{yt}{l_t^\gamma k_t^{1-\gamma}}
\]

Taking the natural log and subsequently applying the laws of logarithms
yields,

\[
log(A_t) = log(yt) - \gamma log(l_t) - (1-\gamma) logk_t
\]

Hence, the previously created variable in our dataset called \(A_t\)
becomes a function of the above equation. Furthermore, we lag \(A_t\)
and now define this new variable as,

\[
\Delta log(A_{t-1}) = \Delta log(yt) - \gamma \Delta log(l_t) - (1-\gamma) \Delta logk_t
\]

Thus, running a regression of \(\Delta log(A_{t})\) on
\(\Delta log(A_{t-1})\), that is, the estimation of the \(AR(1)\)
process yields the \(\rho_{\alpha} = 0.990\) and
\(\sigma_\alpha = 0.008\), as seen in table
\ref{tab:regression_results1}.

\begin{table}[!htbp] \centering 
  \caption{Solow Residual} 
  \label{tab:regression_results1} 
\begin{tabular}{@{\extracolsep{5pt}}lc} 
\\[-1.8ex]\hline 
\hline \\[-1.8ex] 
 & \multicolumn{1}{c}{\textit{Dependent variable:}} \\ 
\cline{2-2} 
\\[-1.8ex] & Productivity process \\ 
\hline \\[-1.8ex] 
 Lag of productivity process & 0.990$^{***}$ \\ 
  & (0.008) \\ 
  Constant & $-$0.023 \\ 
  & (0.019) \\ 
 \hline \\[-1.8ex] 
Observations & 287 \\ 
R$^{2}$ & 0.980 \\ 
Adjusted R$^{2}$ & 0.980 \\ 
Residual Std. Error & 0.067 (df = 285) \\ 
F Statistic & 14,057.840$^{***}$ (df = 1; 285) \\ 
\hline 
\hline \\[-1.8ex] 
\textit{Note:}  & \multicolumn{1}{r}{$^{*}$p$<$0.1; $^{**}$p$<$0.05; $^{***}$p$<$0.01} \\ 
\end{tabular} 
\end{table}

\newpage

\hypertarget{section-6}{%
\section{Section 6}\label{section-6}}

In this section we provide the list of equations that will go into
Dynare software. That is, the equations in \emph{levels} of which we
will let Dynare do a linearisation around the level of the variables.
Using Dynare we will simulate a model with the calibration of parameters
given as:

\[
\beta = 0.97, \ \delta_k = 0.03, \ \alpha = 0.5, \ \gamma = \frac{1}{3}, \ and \ \bar A = 1.
\]

From section 6 we simulated an autoregressive process of order one and
found the persistence and standard deviation of the shock parameters to
be \(\rho_{\alpha} = 0.990\) and \(\sigma_\alpha = 0.008\) respectively.
Hence, the set of equations for the model are:

\begin{align} 
\lambda_t = c_t^{-1} \tag{MU of consumption}
\label{eq:mu_of_consumption}
\end{align}

\begin{align}
\lambda_t = \lambda_{t+1} \beta[r_{t+1} + 1 - \delta] \tag{Euler equation}
\label{eq:intertemporal_margin}
\end{align}

\begin{align}
1 = x_t + l_t \tag{Time constraint}
\label{eq:time_constraint}
\end{align}

\begin{align}
\alpha c_t = x_t w_t \tag{Labour supply}
\label{eq:labour_supply}
\end{align}

\begin{align}
k_t = i_t + (1 - \delta)k_{t-1} \tag{Capital accumulation}
\label{eq:cap_accum}
\end{align}

\begin{align}
y_t = c_t + i_t \tag{Market clearing}
\label{eq:market_clearing}
\end{align}

\begin{align}
y_t = a_t l_t^{\gamma} k_{t-1}^{1 - \gamma}  \tag{Production function}
\label{eq:production_function}
\end{align}

\begin{align}
w_t = \gamma a_t l_t^{\gamma} k_{t-1}^{1 - \gamma} \tag{Wage}
\label{eq:wage}
\end{align}

\begin{align}
r_t = (1 - \gamma) a_t l^{\gamma} k_{t-1}^{- \gamma} \tag{Capital return}
\label{eq:capital_return}
\end{align}

From here, we proceed by reporting the resulting impulse response
function subsequent to a one standard deviation shock, in other words a
positive one percent technology shock from it's steady state level, as
shown in figure \ref{fig11}.

\begin{figure}
    \includegraphics{qD.png}
    \caption{IRFs to a 1\% technology shock in levels}
    \label{fig11}
\end{figure}

From figure \ref{fig11}, we note that a rise in technology translates to
an instant increase in output over the period of 40 periods. We know
that output's two inputs are consumption and investment. Inadvertently,
consumption on the one hand has also risen in line with expectation.
Owing to higher perceived income from economic agents which evidently
rises at a sustained pace, with no indication of return to steady state
level at least of the 40 period time horizon. Employment or rather
labour is inversely related to leisure. With that said, a positive shock
in technology will reduce labour demanded by firms, as a resulted of,
more efficient way of producing cars via usage of advanced machinery for
instance. Revisiting the inverse relationship, we then expect leisure to
rise and labour to fall back toward their steady states after instantly
decreasing and increasing respectively.

There is a substitution effect that prevails from the slow to take off
but eventually rising rental rate of capital as this translates to
lowered discounted price of future consumption. This leads to economic
agents postponing their consumption today to the future. Also, we note
that rising marginal product of capital paired with the rising
consumption will lead to higher investment
(\protect\hyperlink{ref-sims2017}{Sims 2017}). On that note, investment
instantly rises above its steady state level once the technological
shock takes effect, before then embarking on a falling trend over the 40
period horizon. Lastly, there is a instant rise in productivity above
steady state, of which starts to taper off at a slow pace, back toward
its steady state which displays how strong the persistence in the
technology shock is.

\hypertarget{section-7}{%
\section{Section 7}\label{section-7}}

This section is an extension of the previous section in that here we
just apply the logs of the set of equations to our model. Our
persistence and standard deviation of the shock parameters still remain
the same as estimated from table \ref{tab:regression_results1}. Hence,
the set of equations for the model are:

\begin{equation} 
e^{\lambda} = e^{c^{-1}}
\label{eq:mu_of_consumption1}
\end{equation}

\begin{equation}
e^{\lambda} = e^{\lambda_{t+1}} \beta e^{[r_{t+1} + 1 - \delta]}
\label{eq:intertemporal_margin1}
\end{equation}

\begin{equation}
1 = e^{x} + e^{l}
\label{eq:time_constraint1}
\end{equation}

\begin{equation}
\alpha e^{c} = e^{x + w}
\label{eq:labour_supply1}
\end{equation}

\begin{align}
e^{k_t} = e{i_t + (1 - \delta)k_{t-1}}
\label{eq:cap_accum1}
\end{align}

\begin{equation}
e^{y} = e^{c} + e^{i}
\label{eq:market clearing1}
\end{equation}

\begin{equation}
e^{y} = e^{a l^{\gamma} k_{t-1}^{1 - \gamma}}  
\label{eq:production_function1}
\end{equation}

\begin{equation}
e^{w} = \gamma e^{a l^{\gamma} k_{t-1}^{1 - \gamma}}
\label{eq:wage1}
\end{equation}

\begin{equation}
e^{r} = (1 - \gamma)   e^{a l^{\gamma} k_{t-1}^{- \gamma}}
\label{eq:capital_return1}
\end{equation}

\begin{figure}
    \includegraphics{qE.png}
    \caption{The IRFs to a 1\% technology shock in logs}
    \label{fig22}
\end{figure}

From figure \ref{fig22}, we note that output instantly rises, at just
under 0.01\%, above steady state and over the time horizon at an
increasing rate. Then we capture a instant and above steady state rise
as well for consumption. However, the marginal utility of consumption
falls instantly and then continues to embark on the falling trend over
the 40 period time horizon. Hereafter, we note how leisure upon the
shock of technology, begins well below its steady state then for the
rest of the time horizon it edges closer toward its steady state.
Conversely, labour begins well above its steady state and thereafter
tapers down toward its steady state. Investment begins slightly above
0.02\% over its steady state, owing to a positive shock in technology,
then it slowly reduces over the time horizon in the direction of its
steady state. Furthermore, we have the rental rate of capital, which
rises over the time horizon above its steady state. Subsequently, we
note that productivity begins 0.008\% above its steady state, then it
reduces at a slow pace toward its steady state as the technological
shock fades away. Lastly, wages rise over the time horizon, as seen in
figure \ref{fig22}.

\hypertarget{section-8}{%
\section{Section 8}\label{section-8}}

Given the main steady states derived in section 5, we now proceed
expressing the model in terms of log deviations from its deterministic
steady state. This is done by log-linearising the model. The goal with
log-linearisation is to solve the model as an approximation around its
steady state. Hence, we assume that the steady state is a relevant
concept, and look at the dynamics of the model when the economy faces
small departures from steady state.

Furthermore, we log-linearise the full model using the following formula
as a guide:

\begin{equation}
E_{t} [f(X)] \approx E_{t} [f(\bar X) + \frac{\partial f(X)}{\partial X}  X |_{x = \bar x} \hat x]
\label{eq:formula}
\end{equation}

where \(X\) denotes the variable and \(\bar X\) its steady state and
\(log(X) = x\) and \(\hat x\) denotes the log deviation from steady
state, i.e
\(\hat x = log(X) - log(\bar X) = x - \bar x \simeq \frac{x}{ \bar{x}} - 1\).

We begin by log-linearising the production function which yields,

\[
\bar y e^{\hat y_t} = \bar a e^{\hat a_t} (\bar l e^{\hat l_t})^{\gamma} (\bar k e^{\hat k_{t-1}})^{1 - \gamma}
\]

Note in steady state that
\(\bar y = \bar a \bar l^{\gamma} \bar k^{1 - \gamma}\), which
simplifies the above equation to

\[
e^{\hat y_t} = e^{\hat a_t} (e^{\hat l_t})^{\gamma} (e^{\hat k_{t-1}})^{1 - \gamma}
\]

Taking the logs of this, we find the log-linearised production function
equation to yield,

\[
\hat y_t = \hat a_t + \hat l_t^{\gamma}  + \hat k_{t-1}^{1 - \gamma}
\]

Next we follow a similar procedure for the equation of real wage,

\[
\bar w e^{\hat w_t} = \gamma \bar a e^{\hat a_t} [\frac{\bar k e^{\hat k_{t-1}}}{ \bar l e^{\hat l_t}}]^{1 - \gamma}
\]

Again in steady state, we note that
\(\bar w = \gamma \bar a [\frac{\bar k}{\bar l}]^{1- \gamma}\),
simplifies the equation to,

\[
e^{\hat w_t} = e^{\hat a_t} [\frac{e^{\hat k_{t-1}}}{e^{\hat l_t}}]^{1 - \gamma}
\]

Taking the natural log,

\[
log(e^{\hat w_t}) = log(e^{\hat a_t}) + log(e^{\hat l_t})^{1- \gamma} - log(e^{\hat k_{t-1}})^{1- \gamma}
\]

Simplifying this we find ,

\[
\hat w_t = \hat a_t + (1- \gamma)\hat k_{t-1} - (1- \gamma)\hat l_t
\]

The log-linearised real wage equation is,

\[
\hat w_t = \hat a_t + (1- \gamma)[\hat k_{t-1} - \hat l_t]
\]

Next, we find the log-linearised equation for the capital return
otherwise known as the rental rate equation,

\[
\bar r e^{\hat w_t} = (1-\gamma) \bar a e^{\hat a_t} [\frac{\bar k e^{\hat k_{t-1}}}{ \bar l e^{\hat l_t}}]^{\gamma}
\]

Once more in steady state, we note that
\(\bar r = \gamma \bar a [\frac{\bar k}{\bar l}]^{ \gamma}\) which
simplifies the equation to,

\[
e^{\hat r_t} = e^{\hat a_t} [\frac{e^{\hat k_{t-1}}}{e^{\hat l_t}}]^{\gamma}
\]

Taking the natural log, we find the log-linearised capital return to be,

\[
\hat r_t = \hat a_t + \gamma[\hat k_{t-1} - \hat l_t]
\]

Proceeding onward, we solve for the log-linearised capital accumulation
equation as follows,

\[
k_{t+1} = (1-\delta) k_t + i_t
\]

Applying \ref{eq:formula} yields,

\[
\bar k \hat k_{t+1} = (1-\delta) \bar k \hat k_t + \bar i \hat i_t
\]

\[
\hat k_{t+1} = (1-\delta) \hat k_t + \frac{\bar i}{\bar k} \hat i_t
\]

Note in steady state we have \(\bar k = (1-\delta) \bar k + \bar i\) and
also note that \(\frac{\bar i}{\bar k} = \delta\) hence the
log-linearised capital accumulation becomes,

\[
\hat k_{t+1} = (1-\delta) \hat k_t + \delta \hat i_t
\]

Recall the marginal utility of consumption equation
\ref{eq:mu_of_consumption} is:

\[
\lambda_t = c_t^{-1}
\]

Applying equation \ref{eq:formula} on the marginal utility of
consumption yields,

\[
\bar \lambda \hat \lambda = - \frac{1}{(\bar c)^2} \bar c \hat c 
\]

\[
\bar \lambda \hat \lambda = - \frac{1}{\bar c} \hat c 
\]

Note that in steady state \(\bar \lambda = (\bar c)^{-1}\) and hence we
have,

\[
(\bar c)^{-1} \hat \lambda = - \frac{1}{\bar c} \hat c 
\]

As a result, multiplying throughout by \(\bar c\) finally yields,

\[
\hat \lambda = - \hat c
\]

Next we apply equation \ref{eq:formula} on the intertemporal margin
which yields,

\[
\lambda_t = \lambda_{t+1} \beta[r_{t+1} + 1 - \delta]
\]

Next we solve for the log-linearised time constraint equation as
follows,

\[
1 = x_t + l_t
\]

Subsequently applying equation \ref{eq:formula} we obtain,

\[
0 = \bar x \hat x_t + \bar l \hat l_t
\]

Lastly, we solve for the log-linearised labour supply equation as
follows,

\[
\alpha c_t = x_t w_t
\]

Applying equation \ref{eq:formula} yields,

\[
\alpha \bar c \hat c_t = \bar x \hat x_t + \bar w \hat w_t
\]

We note in steady state that the labour supply is
\(\alpha \bar c = \bar x \bar w\). Hence our log-linearised labour
supply is,

\[
\hat c_t =  \hat x_t + \hat w_t
\]

For our log-linearised market clearing equation we have,

\[
\bar y \hat y_t = \bar c \hat c_t + \bar w \hat w_t
\]

Which can also be written as

\[
\hat y_t = \frac{\bar c}{\bar y} \hat c_t + \frac{\bar w}{\bar y} \hat w_t
\]

Thus the full set of log-linearised equations are given as:

\begin{align} 
\hat \lambda = - \hat c \tag{MU of consumption}
\label{eq:mu_of_consumption2}
\end{align}

\begin{align}
\hat \lambda_t = \hat \lambda_{t+1} + \hat r_{t+1} [1- \beta (1 - \delta)] \tag{Euler equation}
\label{eq:intertemporal_margin2}
\end{align}

\begin{align}
0 = (\frac{\bar x}{\bar x + \bar l} ) \hat x_t + (\frac{\bar l}{\bar x + \bar l} ) \hat l_t \tag{Time constraint}
\label{eq:time_constraint2}
\end{align}

\begin{align}
\hat c_t =  \hat x_t + \hat w_t \tag{Labour supply}
\label{eq:labour_supply2}
\end{align}

\begin{align}
\hat k_{t} = (1-\delta) \hat k_{t+1} + \delta \hat i_t \tag{Capital accumulation}
\label{eq:cap_accum2}
\end{align}

\begin{align}
\hat y_t = \frac{\bar c}{\bar y} \hat c_t + \frac{\bar w}{\bar y} \hat w_t \tag{Market clearing}
\label{eq:market_clearing2}
\end{align}

\begin{align}
\hat y_t = \hat a_t + \hat l_t^{\gamma}  + \hat k_{t-1}^{1 - \gamma}  \tag{Production function}
\label{eq:production_function2}
\end{align}

\begin{align}
\hat w_t = \hat a_t + (1- \gamma)[\hat k_{t-1} - \hat l_t] \tag{Wage}
\label{eq:wage2}
\end{align}

\begin{align}
\hat r_t = \hat a_t + \gamma[\hat k_{t-1} - \hat l_t] \tag{Capital return}
\label{eq:capital_return2}
\end{align}

\hypertarget{section-9}{%
\section{Section 9}\label{section-9}}

In this last section of the paper, we provide the graphical illustration
of the log-linearised IRF in response to the 1\% technology shock.
Thereafter, briefly discuss if this result should match the one from
sections 6 and 7 respectively.

\begin{figure}
    \includegraphics{qH.png}
    \caption{IRFs to a 1\% technology shock in log-linearisation}
    \label{fig3}
\end{figure}

The 1\% impulse in technology on all the other macroeconomic variables
generally seems to have the desired effect. In \ref{fig3}, output
continues to rise over the 40 period horizon with the rise of just under
0.01\% being similar to that from \ref{fig22}. Investment declines rises
just above 0.02\% in response to a 1\% shock in technology of which it
then declines over the 40 period time horizon toward its steady state.
The rise in investment leads to a subsequent rise in consumption, of
which it continues to rise with no sign of reverting back to their
steady state levels over the 40 period time horizon. Since we expect
economic agents to smooth their consumption over time, owing to the
positive wealth effect and perceived higher income, we note as the
reason why consumption is persistently high. However, as the
technological shock dies out they will return back to their steady
state.

Owing to the inverse relationship between labour and leisure. A 1\%
shock in technology results in a increase in labour supply and
conversely a reduction in leisure over the 40 period horizon with both
returning to their steady state levels overtime. At the initial
technology shock, the rental rate of capital marginally rises at just
above 0\% before embarking on a rising trend, with no indication of of
returning to its steady state level, over the time horizon. On the other
hand, wages maintains a increase within the bounds of 0.005\% and
0.015\% over the time horizon, with no indication of returning back to
its steady state level as well. Lastly, the technology shock seems
fairly persistent, falling at a slow pace over the time horizon, as seen
on the productivity IRF in figure \ref{fig3}.

From figure \ref{fig3}, the shape of the IRFs of the levels, logs as
well as the log-linearised models take on the same shape. However, we
note that the results for the log model IRFs and the log-linearised
model IRFs match, both of which are different from the level model IRF.
In addition, the results are as expected since we did not account for
higher order terms in the Taylor expansion. Hence, we obtained a system
of linear stochastic difference equations that we could solve for
(\protect\hyperlink{ref-harris2015}{Harris 2015}). Moreover, for most of
the equations post the log-linearisation procedure, we found the
equations to be similar to their original equations or non
log-linearised forms.

\newpage

\hypertarget{conclusion}{%
\section{Conclusion}\label{conclusion}}

In this study, we built a benchmark model by making use of micro-founded
methodology to model how various macroeconomic outcomes would be
effected post a 1\% shock in the technology stock of an economy. We
decentralised the representative agent problem for both the consumer and
the firm, of which prices (wages and interest) became explicitly
defined. A further level of nuance in the model was including not only
the capital stock state variable but an additional state variable for
bonds.

Moreover, our evidence on a 1\% shock was modeled in two folds. Firstly,
it was modeled in levels, then in logs and we also made use of a
log-linearisation technique. Overall, the shape of the responses of the
various macroeconomic variables were all of the same resemblance.
However, the interpretation was slightly different across all three
categories with respect to the magnitude of the deviation away from the
steady state. Since we higher order terms was neglected when
log-linearising the model, we found the model in logs and the model
log-linearisation were similar by shape and magnitude. However, the log
and log-linearised were different in magnitude to when the shock in the
model equation, that is the autoregressive order one process was just
defined in levels.

Given the high level of persistence we found in table
\ref{tab:regression_results1}, we note the model display that
persistence with the shock in technology taking longer to die out, since
the technological shock is only transitory. In other words, the
macroeconomic variables in figures \ref{fig11}, \ref{fig22} and
\ref{fig3} take prolonged time to return return to steady state. This
DGE model was built using Dynare, which is an extension MATLAB.
Nonetheless, the economic meaning across all variables in all three
models was consistent and in line with expectation. The findings were
that output, consumption, investment, labour, the rental rate of capital
and wages increased after a 1\% shock in technology. Whereas, the
marginal utility of consumption and leisure subsided after a 1\% shock
in technology. Overall, this paper explored on the way current
macroeconomic analysis is done at the frontier of macroeconomic research
and at central banks and government.

\newpage

\hypertarget{appendix}{%
\section{Appendix}\label{appendix}}

In this section, we present the code to build the time series for the
Total Factor Productivity shock as well as the Dynare code for the IRFs.

\begin{verbatim}
## The following objects are masked from data1:
## 
##     Capital, Date, gamma, Labour, lCapital, lGDP, lLabour, Price index,
##     Real_GDP
\end{verbatim}

\begin{table}[!htbp] \centering 
  \caption{Solow Residual} 
  \label{tab:regression_results} 
\begin{tabular}{@{\extracolsep{5pt}}lc} 
\\[-1.8ex]\hline 
\hline \\[-1.8ex] 
 & \multicolumn{1}{c}{\textit{Dependent variable:}} \\ 
\cline{2-2} 
\\[-1.8ex] & Productivity process \\ 
\hline \\[-1.8ex] 
 Lag of productivity process & 0.990$^{***}$ \\ 
  & (0.008) \\ 
  Constant & $-$0.023 \\ 
  & (0.019) \\ 
 \hline \\[-1.8ex] 
Observations & 287 \\ 
R$^{2}$ & 0.980 \\ 
Adjusted R$^{2}$ & 0.980 \\ 
Residual Std. Error & 0.067 (df = 285) \\ 
F Statistic & 14,057.840$^{***}$ (df = 1; 285) \\ 
\hline 
\hline \\[-1.8ex] 
\textit{Note:}  & \multicolumn{1}{r}{$^{*}$p$<$0.1; $^{**}$p$<$0.05; $^{***}$p$<$0.01} \\ 
\end{tabular} 
\end{table}

\begin{figure}
    \includegraphics{level.png}
    \caption{Model in levels}
    \label{fig4}
\end{figure}

\begin{figure}
    \includegraphics{log.png}
    \caption{Model in logs}
    \label{fig5}
\end{figure}

\begin{figure}
    \includegraphics{linearisation.png}
    \caption{Model in log-linearisation}
    \label{fig6}
\end{figure}

\newpage

\hypertarget{references}{%
\section*{References}\label{references}}
\addcontentsline{toc}{section}{References}

\hypertarget{refs}{}
\begin{CSLReferences}{1}{0}
\leavevmode\vadjust pre{\hypertarget{ref-harris2015}{}}%
Harris, D. 2015. {``The Real Business Cycle Model.''}
\url{http://harrisdellas.net/teaching/semmacro15/rbc.pdf}.

\leavevmode\vadjust pre{\hypertarget{ref-sims2017}{}}%
Sims, Eric. 2017. {``Graduate Macro Theory II: The Real Business Cycle
Model.''} University of Notre Dame; Class Lecture Notes.

\leavevmode\vadjust pre{\hypertarget{ref-solow1957technical}{}}%
Solow, Robert M. 1957. {``Technical Change and the Aggregate Production
Function.''} \emph{The Review of Economics and Statistics} 39 (3):
312--20.

\end{CSLReferences}

\end{document}
